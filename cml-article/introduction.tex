\section{Introduction}
%
In order to address the challenges of the ever-changing, increasingly distributed technologies used on software systems, the Model-Driven Architecture (MDA \cite{mda}) initiative by the Object Management Group (OMG) has been promoting model-driven software development.
In particular, MDA has guided the use of high-level models (created with OMG standards, such as UML \cite{uml}, OCL \cite{ocl} and MOF \cite{mof}) to derive software artifacts and implementations via automated transformations.
As one of its value propositions, the MDA guide \cite{mda} advocates:
\begin{quote}``Automation reduces the time and cost of realizing a design, reduces the time and cost for changes and maintenance and produces results that ensure consistency across all of the derived artifacts. For example, manually producing all of the web service artifacts required to implement a set of processes and services for an organization is difficult and error-prone. Producing execution artifacts from a model is more reliable and faster.''\end{quote} 

The Metaprogramming System (MPS \cite{voelter}) has shown that tooling can assist in integrating high-level, domain-specific models in the development workflow.

